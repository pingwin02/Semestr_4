\documentclass{article}
\usepackage[T1]{fontenc}
\usepackage{lmodern}
\usepackage[polish]{babel}
\usepackage{graphicx}
\usepackage{float}
\usepackage{amsmath}
\usepackage{hyperref}

\usepackage[a4paper, margin=2.54cm]{geometry}


\title{Sprawozdanie - Projekt 2\\Układy równań liniowych\\
Implementacja metod Jacobiego, Gaussa-Seidla i LU}
\author{Damian Jankowski s188597}
\date{27 kwietnia 2023}

\begin{document}

\maketitle

\section{Wstęp}
Celem projektu było zaimplementowanie metod Jacobiego, 
Gaussa-Seidla i LU oraz porównanie wydajności,
dokładności, jak również czasu ich wykonania.

Każda z metod w różny sposób rozwiązuje pewien układ równań liniowych:

\begin{equation} 
    \boldsymbol{Ax = b}
\end{equation} 
gdzie: 
\begin{itemize}
    \item $\boldsymbol{A}$ -- macierz kwadratowa zawierająca współczynniki układu równań,
    \item $\boldsymbol{b}$ -- wektor wyrazów wolnych, 
    \item $\boldsymbol{x}$ -- wektor rozwiązań układu.
\end{itemize}

Macierz $A$ została zdefiniowana jako macierz pasmowa o rozmiarze $997 \times 997$:

\begin{equation} \label{eq:macierzA}
A = \begin{bmatrix}
    a_1 & a_2 & a_3 & 0 & 0 & 0 & 0 & \dots & 0\\
    a_2 & a_1 & a_2 & a_3 & 0 & 0 & 0 & \dots & 0\\
    a_3 & a_2 & a_1 & a_2 & a_3 & 0 & 0 & \dots & 0\\
    0 & a_3 & a_2 & a_1 & a_2 & a_3 & 0 & \dots & 0\\
    0 & 0 & a_3 & a_2 & a_1 & a_2 & a_3 & \dots & 0\\
    0 & 0 & 0 & a_3 & a_2 & a_1 & a_2 & \dots & 0\\
    0 & 0 & 0 & 0 & a_3 & a_2 & a_1 & \dots & 0\\
    \vdots & \vdots & \vdots & \vdots & \vdots & \vdots & \vdots & \ddots & \vdots\\
    0 & 0 & \dots & 0 & 0 & 0 & a_3 & a_2 & a_1
\end{bmatrix}
\end{equation}
gdzie:
\begin{equation*}
a_1 = 10, \quad a_2 = -1, \quad a_3 = -1
\end{equation*}

Wektor $b$ długości $997$ został zdefiniowany jako:

\begin{equation}
b = \begin{bmatrix}
    sin(0 \cdot (f + 1))\\
    sin(1 \cdot (f + 1))\\
    sin(2 \cdot (f + 1))\\
    \vdots\\
    sin(996 \cdot (f + 1))
\end{bmatrix} \quad f = 8
\end{equation}

\section{Metody rozwiązywania układów równań liniowych}
Istnieje wiele metod rozwiązywania układów równań liniowych. 
W projekcie zostały zaimplementowane trzy z nich: metoda Jacobiego, Gaussa-Seidla i LU.

Pierwsze dwa należą do grupy metod iteracyjnych, 
natomiast ostatnia jest metodą bezpośrednią.

\subsection{Metody iteracyjne}
Metody iteracyjne polegają na wyznaczeniu 
kolejnych przybliżeń rozwiązania układu równań liniowych.

Korzystają one z tzw. macierzy: trójkątnej dolnej (\textit{Lower}) $L$, 
górnej (\textit{Upper}) $U$ oraz diagonalnej $D$,
które spełniają warunek:
\begin{equation}
    \boldsymbol{A = L + U + D}
\end{equation}

Przykładowo dla macierzy $A$ \eqref{eq:macierzA},
macierze $L$, $U$ i $D$ wyglądają następująco:

\begin{equation}
L = \begin{bmatrix}
    0 & 0 & 0 & 0 & 0 & 0 & 0 & \dots & 0\\
    a_2 & 0 & 0 & 0 & 0 & 0 & 0 & \dots & 0\\
    a_3 & a_2 & 0 & 0 & 0 & 0 & 0 & \dots & 0\\
    0 & a_3 & a_2 & 0 & 0 & 0 & 0 & \dots & 0\\
    0 & 0 & a_3 & a_2 & 0 & 0 & 0 & \dots & 0\\
    0 & 0 & 0 & a_3 & a_2 & 0 & 0 & \dots & 0\\
    0 & 0 & 0 & 0 & a_3 & a_2 & 0 & \dots & 0\\
    \vdots & \vdots & \vdots & \vdots & \vdots & \vdots & \vdots & \ddots & \vdots\\
    0 & 0 & \dots & 0 & 0 & 0 & a_3 & a_2 & 0
\end{bmatrix}
\end{equation}

\begin{equation}
U = \begin{bmatrix}
    0 & a_2 & a_3 & 0 & 0 & 0 & 0 & \dots & 0\\
    0 & 0 & a_2 & a_3 & 0 & 0 & 0 & \dots & 0\\
    0 & 0 & 0 & a_2 & a_3 & 0 & 0 & \dots & 0\\
    0 & 0 & 0 & 0 & a_2 & a_3 & 0 & \dots & 0\\
    0 & 0 & 0 & 0 & 0 & a_2 & a_3 & \dots & 0\\
    0 & 0 & 0 & 0 & 0 & 0 & a_2 & \dots & 0\\
    0 & 0 & 0 & 0 & 0 & 0 & 0 & \dots & 0\\
    \vdots & \vdots & \vdots & \vdots & \vdots & \vdots & \vdots & \ddots & \vdots\\
    0 & 0 & \dots & 0 & 0 & 0 & 0 & 0 & 0
\end{bmatrix}
\end{equation}

\begin{equation}
D = \begin{bmatrix}
    a_1 & 0 & 0 & 0 & 0 & 0 & 0 & \dots & 0\\
    0 & a_1 & 0 & 0 & 0 & 0 & 0 & \dots & 0\\
    0 & 0 & a_1 & 0 & 0 & 0 & 0 & \dots & 0\\
    0 & 0 & 0 & a_1 & 0 & 0 & 0 & \dots & 0\\
    0 & 0 & 0 & 0 & a_1 & 0 & 0 & \dots & 0\\
    0 & 0 & 0 & 0 & 0 & a_1 & 0 & \dots & 0\\
    0 & 0 & 0 & 0 & 0 & 0 & a_1 & \dots & 0\\
    \vdots & \vdots & \vdots & \vdots & \vdots & \vdots & \vdots & \ddots & \vdots\\
    0 & 0 & \dots & 0 & 0 & 0 & 0 & 0 & a_1
\end{bmatrix}
\end{equation}


Warunkiem zakończenia iteracji jest osiągnięcie
zadanej dokładności lub maksymalnej liczby iteracji.
\subsubsection{Metoda Jacobiego}

Metoda Jacobiego polega na wyznaczeniu kolejnych przybliżeń
rozwiązania układu równań liniowych zgodnie ze wzorem:

\begin{equation}
    \boldsymbol{x^{(k+1)} = -D^{-1}(L + U)x^{(k)} + D^{-1}b}
\end{equation}
gdzie:
\begin{itemize}
    \item $\boldsymbol{x^{(k)}}$ - wektor przybliżenia rozwiązania w $k$-tej iteracji
\end{itemize}

Problemem tej metody jest konieczność wyznaczenia odwrotności macierzy $D$,
co może być bardzo kosztowne obliczeniowo. Natomiast z racji, że
macierz $D$ jest diagonalna, to jej odwrotność jest równa odwrotności
każdego z jej elementów na przekątnej.

\subsubsection{Metoda Gaussa-Seidla}

Metoda Gaussa-Seidla polega na wyznaczeniu kolejnych przybliżeń
rozwiązania układu równań liniowych zgodnie ze wzorem:

\begin{equation}
    \boldsymbol{x^{(k+1)} = -D^{-1}(L + U)x^{(k+1)} + D^{-1}b}
\end{equation}




\end{document}
\documentclass{article}
\usepackage[T1]{fontenc}
\usepackage{lmodern}
\usepackage[polish]{babel}
\usepackage{graphicx}
\usepackage{amsmath}
\usepackage{hyperref}
\usepackage{listings}
\usepackage{xcolor}
\usepackage{float}

\lstdefinestyle{mystyle}{
    language=C,
    keywordstyle=\color{blue},
    identifierstyle=\color{teal},
    stringstyle=\color{red},
    commentstyle=\color{gray},
    showstringspaces=false,
    frame=single
}

\lstset{style=mystyle}

\usepackage[a4paper, margin=2.54cm]{geometry}


\title{Sprawozdanie - Projekt 2\\Układy równań liniowych\\
Implementacja metod Jacobiego, Gaussa-Seidla i LU}
\author{Damian Jankowski s188597}
\date{27 kwietnia 2023}

\begin{document}

\maketitle

\tableofcontents

\section{Wstęp}
Celem projektu było zaimplementowanie metod Jacobiego, 
Gaussa-Seidla i LU oraz porównanie wydajności,
dokładności, jak również czasu ich wykonania.

Każda z metod w różny sposób rozwiązuje pewien układ równań liniowych:

\begin{equation} 
    \boldsymbol{Ax = b}
\end{equation} 
gdzie: 
\begin{itemize}
    \item $\boldsymbol{A}$ -- macierz kwadratowa zawierająca współczynniki układu równań,
    \item $\boldsymbol{b}$ -- wektor wyrazów wolnych, 
    \item $\boldsymbol{x}$ -- wektor rozwiązań układu.
\end{itemize}

Macierz $A$ została zdefiniowana jako macierz pasmowa o rozmiarze $997 \times 997$:

\begin{equation} \label{eq:macierzA}
A = \begin{bmatrix}
    a_1 & a_2 & a_3 & 0 & 0 & 0 & 0 & \dots & 0\\
    a_2 & a_1 & a_2 & a_3 & 0 & 0 & 0 & \dots & 0\\
    a_3 & a_2 & a_1 & a_2 & a_3 & 0 & 0 & \dots & 0\\
    0 & a_3 & a_2 & a_1 & a_2 & a_3 & 0 & \dots & 0\\
    0 & 0 & a_3 & a_2 & a_1 & a_2 & a_3 & \dots & 0\\
    0 & 0 & 0 & a_3 & a_2 & a_1 & a_2 & \dots & 0\\
    0 & 0 & 0 & 0 & a_3 & a_2 & a_1 & \dots & 0\\
    \vdots & \vdots & \vdots & \vdots & \vdots & \vdots & \vdots & \ddots & \vdots\\
    0 & 0 & \dots & 0 & 0 & 0 & a_3 & a_2 & a_1
\end{bmatrix}
\end{equation}
gdzie:
\begin{equation*}
a_1 = 10, \quad a_2 = -1, \quad a_3 = -1
\end{equation*}

Wektor $b$ długości $997$ został zdefiniowany jako:

\begin{equation}
b = \begin{bmatrix}
    sin(0 \cdot (f + 1))\\
    sin(1 \cdot (f + 1))\\
    sin(2 \cdot (f + 1))\\
    \vdots\\
    sin(996 \cdot (f + 1))
\end{bmatrix} \quad f = 8
\end{equation}

\section{Metody rozwiązywania układów równań liniowych}
Istnieje wiele metod rozwiązywania układów równań liniowych. 
W projekcie zostały zaimplementowane trzy z nich: metoda Jacobiego, Gaussa-Seidla i LU.

Pierwsze dwa należą do grupy metod iteracyjnych, 
natomiast ostatnia jest metodą bezpośrednią.

\subsection{Metody iteracyjne}
Metody iteracyjne polegają na wyznaczeniu 
kolejnych przybliżeń rozwiązania układu równań liniowych.

Korzystają one z tzw. macierzy: trójkątnej dolnej (\textit{Lower}) $L$, 
górnej (\textit{Upper}) $U$ oraz diagonalnej $D$,
które spełniają warunek:
\begin{equation}
    \boldsymbol{A = L + U + D}
\end{equation}

Warunkiem zakończenia iteracji jest osiągnięcie
zadanej dokładności lub maksymalnej liczby iteracji.
\subsubsection{Metoda Jacobiego}

Metoda Jacobiego polega na wyznaczeniu kolejnych przybliżeń
rozwiązania układu równań liniowych zgodnie ze wzorem:

\begin{equation}
    \boldsymbol{x^{(k+1)} = -D^{-1}(L + U)x^{(k)} + D^{-1}b}
\end{equation}
gdzie:
\begin{itemize}
    \item $\boldsymbol{x^{(k)}}$ - wektor przybliżenia rozwiązania w $k$-tej iteracji
\end{itemize}

\subsubsection{Metoda Gaussa-Seidla}

Metoda Gaussa-Seidla podobnie jak metoda Jacobiego polega 
na wyznaczeniu kolejnych przybliżeń, jednakże zgodnie z tym wzorem:

\begin{equation}
    \boldsymbol{x^{(k+1)} = -(D+L)^{-1}Ux^{(k)} + (D+L)^{-1}b}
\end{equation}

Problemem tej metody jest konieczność wyznaczenia macierzy $(D+L)^{-1}$,
czego powinno się unikać z racji dużej złożoności obliczeniowej.

Z tego powodu zamiast wyznaczać odwrotność macierzy $D+L$,
stosuje się tzw. podstawienie w przód (ang. \textit{forward substitution}).

\subsubsection{Podstawienie w przód}
Metoda podstawienia w przód polega na wyznaczeniu kolejnych
wartości wektora rozwiązań $\boldsymbol{x}$ układu równań $\boldsymbol{Lx = b}$,
w następujący sposób:

\begin{equation*}
    \boldsymbol{x_1 = \frac{b_1}{l_{11}}}
\end{equation*}

\begin{equation*}
    \boldsymbol{x_2 = \frac{b_2 - l_{21}x_1}{l_{22}}}
\end{equation*}

\begin{equation*}
    \boldsymbol{x_i = \frac{b_i - \sum_{j=1}^{i-1}l_{ij}x_j}{l_{ii}}}
\end{equation*}

\begin{equation*}
    \boldsymbol{\vdots}
\end{equation*}

\begin{equation*}
    \boldsymbol{x_n = \frac{b_n - \sum_{j=1}^{n-1}l_{nj}x_j}{l_{nn}}}
\end{equation*}

Koniecznym jest by macierz $\boldsymbol{L}$ była macierzą trójkątną dolną,
np. w przypadku sumy macierzy $D+L$.

\subsubsection{Podstawienie w tył}
Metoda podstawienia w tył podobnie jak metoda podstawienia w przód
polega na wyznaczeniu kolejnych wartości wektora rozwiązań $\boldsymbol{x}$ tym razem
układu równań $\boldsymbol{Ux = b}$. Jednakże w tym przypadku koniecznym jest
by macierz $\boldsymbol{U}$ była macierzą trójkątną górną.
Kolejne kroki wyglądają następująco:

\begin{equation*}
    \boldsymbol{x_n = \frac{b_n}{u_{nn}}}
\end{equation*}

\begin{equation*}
    \boldsymbol{x_{n-1} = \frac{b_{n-1} - u_{n-1,n}x_n}{u_{n-1,n-1}}}
\end{equation*}

\begin{equation*}
    \boldsymbol{x_i = \frac{b_i - \sum_{j=i+1}^{n}u_{ij}x_j}{u_{ii}}}
\end{equation*}

\begin{equation*}
    \boldsymbol{\vdots}
\end{equation*}

\begin{equation*}
    \boldsymbol{x_1 = \frac{b_1 - \sum_{j=2}^{n}u_{1j}x_j}{u_{11}}}
\end{equation*}

\subsubsection{Warunek zakończenia}
By sprawdzić czy osiągnięto zadaną dokładność należy przy każdej iteracji
sprawdzać czy norma tzw. wektora residuum $\boldsymbol{res}$ jest mniejsza 
od zadanej wartości, np. $10^{-9}$.

Wektor residuum jest zdefiniowany następująco:
\begin{equation}
    \boldsymbol{res^{(k)} = Ax^{(k)}-b}
\end{equation}
W idealnej sytuacji powinien być równy wektorowi zerowemu.

Natomiast w większości przypadków, by sprawdzić czy osiągnięto zadaną dokładność
wyznacza się normę wektora residuum $\boldsymbol{res}$:

\begin{equation}
    \boldsymbol{||res^{(k)}|| = \sqrt{\sum_{i=1}^{n}(res_i^{(k)})^2}}
\end{equation}

Tym sposobem możemy z góry określić dokładność rozwiązania.

\subsection{Metody bezpośrednie}
Metody bezpośrednie polegają na wyznaczeniu rozwiązania układu równań
bezpośrednio z macierzy współczynników $\boldsymbol{A}$. Odznaczają się one
dużą dokładnością, jednakże są czasochłonne i zasobożerne.

\subsubsection{Metoda faktoryzacji LU}

Metoda faktoryzacji LU polega na rozkładzie macierzy 
współczynników $\boldsymbol{A}$
na iloczyn macierzy $\boldsymbol{L}$ i $\boldsymbol{U}$:

\begin{equation}
    \boldsymbol{A = LU}
\end{equation}
gdzie:
\begin{itemize}
    \item $\boldsymbol{L}$ - macierz trójkątna dolna
    \item $\boldsymbol{U}$ - macierz trójkątna górna
\end{itemize}

Wtedy układ równań $\boldsymbol{Ax = b}$ można zapisać jako:

\begin{equation*}
    \boldsymbol{LUx = b}
\end{equation*}

By przejść dalej konieczne jest wyznaczenie potrzebnych macierzy pomocniczych.
\begin{enumerate}
    \item Na początku tworzy się macierz $\boldsymbol{L}$, która jest macierzą 
    jednostkową, czyli taką, której elementy na głównej przekątnej są równe 1.
    Natomiast macierz $\boldsymbol{U}$ to kopia macierzy $\boldsymbol{A}$.
    
    Faktoryzację LU opisać można tym kodem w języku C:
    
    \begin{lstlisting}
        for (int k = 0; k < n - 1; k++) {
            for (int j = k + 1; j < n; j++) {
                L[j][k] = U[j][k] / U[k][k];
                for (int i = k; i < n; i++) {
                    U[j][i] = U[j][i] - L[j][k] * U[k][i];
                }
            }
        }
    \end{lstlisting}

    \item Następnie metodą podstawiania w przód wyznacza 
    się wektor $\boldsymbol{y}$,
    który jest rozwiązaniem układu równań $\boldsymbol{Ly = b}$.

    \item Ostatnim krokiem jest wyznaczenie wektora rozwiązań $\boldsymbol{x}$,
    który jest rozwiązaniem układu równań $\boldsymbol{Ux = y}$.
    Z racji, że macierz $\boldsymbol{U}$ jest macierzą trójkątną górną,
    to wyznaczenie wektora $\boldsymbol{x}$ jest możliwe metodą 
    podstawienia w tył.
\end{enumerate}

\section{Implementacja i analiza wyników}
\subsection{Zadanie A}
Zadanie A polegało na zaimplementowaniu układu równań przedstawionego we
wstępie.
\subsection{Zadanie B}
Zadanie B polegało na zaimplementowaniu metod iteracyjnych Jasobiego i 
Gaussa-Seidla, sprawdzeniu ilość iteracji potrzebnych do zakończenia.

Wyniki zostały przedstawione poniżej.

\begin{center}
    \begin{tabular}{| c | c | c |} 
    \hline
    \multicolumn{3}{|c|}{\shortstack{Tabela 1. Porównanie czasu \\ i ilości iteracji
    dla zadania A}} \\
    \hline
    Metoda & Czas & Ilość iteracji \\ [0.5ex] 
    \hline
    Jacobi & 0.12s & 18 \\
    \hline
    Gauss-Seidel & 0.094s & 14 \\
    \hline
    \end{tabular}
\end{center}


\begin{figure}[H]
    \includegraphics[width=0.75\textwidth]{zadB.png}
    \centering
    \caption{Wykres przedstawiający wartość normy wektora residuum od iteracji}
\end{figure}


\end{document}
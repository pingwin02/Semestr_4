\documentclass{article}
\usepackage[T1]{fontenc}
\usepackage{lmodern}
\usepackage[polish]{babel}
\usepackage{graphicx}
\usepackage{float}
\usepackage{hyperref}

\usepackage[a4paper, margin=2.54cm]{geometry}


\title{Sprawozdanie - Projekt 1\\Wskaźnik giełdowy MACD}
\author{Damian Jankowski s188597}
\date{06 kwietnia 2023}

\begin{document}

    \maketitle

    \section{Wstęp}
    \subsection{Czym jest wskaźnik MACD?}
    Wskaźnik \textbf{MACD} (ang. \textit{Moving Average Convergence Divergence}) 
    jest jednym z najpopularniejszych wskaźników giełdowych, badającym
    zależności pomiędzy krótkimi i długimi średnimi ruchomymi.
    Korzystają z niego zarówno inwestorzy indywidualni, 
    jak i instytucjonalni, a także banki i inwestycyjne domy maklerskie.

    \subsection{Jak wyznaczać MACD?}
    Wskaźnik kreślony jest za pomocą dwóch wykresów:
    \begin{itemize}
        \item \textbf{MACD}
        \item \textbf{SIGNAL}
    \end{itemize}

    Obie z nich wykorzystują tzw. wykładniczą średnią ruchomą/kroczącą
    \textbf{EMA} (ang. \textit{Exponential Moving Average}), którą da się opisać
    następującą formułą:
    \begin{equation}
        EMA_{N} = \frac{p_{0} + (1-\alpha)p_{1} + (1-\alpha)^2p_{2} + 
        (1-\alpha)^3p_{3} + ... + (1-\alpha)^Np_{N} }
        {1 + (1-\alpha) + (1-\alpha)^2 + (1-\alpha)^3 + ... + (1-\alpha)^N}
    \end{equation}
    gdzie:
    \begin{itemize}
        \item $EMA_{N}$ - wykładnicza średnia ruchoma z okresu $N$
        \item $p_{0}$ - aktualna cena akcji
        \item $p_{1}$ - cena akcji z poprzedniego okresu
        \item $p_{N}$ - cena akcji z $N$-tego okresu
        \item $\alpha = \frac{2}{N+1}$ - współczynnik wykładniczy
        \item $N$ - liczba okresów wykładniczej średniej ruchomej
    \end{itemize}

    Wartość \textbf{MACD} jest wyliczana na podstawie następującej formuły:
    \begin{equation}
        MACD = EMA_{12} - EMA_{26}
    \end{equation}

    Natomiast \textbf{SIGNAL} to wykładnicza średnia ruchoma z okresu 9, policzona z \textbf{MACD}:
    \begin{equation}
        SIGNAL = EMA_{9}(MACD)
    \end{equation}

    \subsection{Jak interpretować MACD?}
    Przyjęto następujące zasady interpretacji:
    \begin{enumerate}
        \item Jeśli \textbf{MACD} przekroczy \textbf{SIGNAL} z dołu oznacza, że akcje powinny być kupowane.
        \item Jeśli \textbf{MACD} przekroczy \textbf{SIGNAL} z góry oznacza, że akcje powinny być sprzedawane.
    \end{enumerate}

    \section{Analiza wskaźnika MACD}
    \subsection{Cel analizy}
    Celem analizy było sprawdzenie, czy wskaźnik MACD jest dobrym wskaźnikiem do inwestowania w akcje
    oraz czy jest on w stanie przewidywać zmiany cen akcji w zależności od sytuacji na rynku.
    \subsection{Dane wejściowe}
    Dane wejściowe zostały pobrane z serwisu \textbf{Yahoo! Finance API} w formacie \textit{.csv} i obejmują
    ceny akcji firmy \textbf{CD Projekt Red} od 2019-01-16 do 2022-11-30.
    Był to dość ciężki okres, gdyż w tym czasie wydano grę
    \textbf{Cyberpunk 2077}, która posiadała wielkie problemy z optymalizacją podczas premiery,
    co znacząco wpłynęło na rynek.

    \begin{figure}[H]
        \includegraphics[width=\textwidth]{wykres.png}
        \centering
        \caption{Wykres ceny akcji firmy CD Projekt Red}
    \end{figure}

    Dla wybranego przedziału czasu istnieje dokładnie 1000 próbek.

    \subsection{Implementacja}
    W celu implementacji wskaźnika został napisany skrypt w języku \textit{Python}.
    Znajduje się on w pliku \textit{main.py} oraz \textit{functions.py}. Do tworzenia wykresów
    został użyty moduł \textit{matplotlib}, natomiast do wczytywania danych z API --- moduł \textit{pandas}.
    \subsection{Symulacja}
    W celu sprawdzenia poprawności działania wskaźnika została przeprowadzona symulacja.
    Polegała ona na kupowaniu i sprzedawaniu akcji w momencie, gdy \textbf{MACD} przekroczył \textbf{SIGNAL}.
    Zaczynało się z tysiącem złotych oraz zerową ilością akcji.
    Zgodnie z przyjętymi założeniami w momencie, gdy \textbf{MACD} przekroczył \textbf{SIGNAL} z dołu, kupowano maksymalnie możliwą ilość akcji 
    za ówczesną cenę.
    Natomiast, gdy \textbf{MACD} przekroczył \textbf{SIGNAL} z góry, sprzedawano wszystkie akcje.
    
    \begin{figure}[H]
        \includegraphics[width=\textwidth]{macdsignal.png}
        \centering
        \caption{Wykres \textbf{MACD} oraz \textbf{SIGNAL} dla przyjętego okresu czasu}
    \end{figure}

    \begin{figure}[H]
        \includegraphics[width=\textwidth]{simulation.png}
        \centering
        \caption{Wykres ilości gotówki w trakcie symulacji}
    \end{figure}

    W trakcie symulacji algorytm drukował informacje o aktualnie podejmowanej decyzji.
    Dla wybranej polityki inwestycyjnej, po zakończeniu symulacji, algorytm zanotował następujący zysk:
    \begin{verbatim}
        ...
        Dzień:  2022-11-02 00:00:00 Kupiłem akcje za 1089.760008 PLN.
        Dzień:  2022-11-21 00:00:00 Sprzedałem akcje za 1114.239992 PLN
        Zysk: 151.0900469999999 PLN. Pozostało akcji: 0 sztuk.
        Sprzedaję resztę akcji za 0.0 PLN.
        Zysk całkowity: 151.0900469999999 PLN.
    \end{verbatim}

    
    \subsection{Wnioski}
    Gdy przyjrzeć się fragmentowi wykresu \textbf{MACD} oraz \textbf{SIGNAL}, można zauważyć, że
    algorytm podejmował dobre decyzje. Natomiast patrząc na cały okres czasu, zysk mógłby być
    znacznie większy, gdyby algorytm nie podejmował decyzji zbyt późno.

    \begin{figure}[H]
        \includegraphics[width=\textwidth]{macdsignal_zoom.png}
        \centering
        \caption{Fragment wykresu \textbf{MACD} oraz \textbf{SIGNAL}}{ wraz z zaznaczonymi punktami kupna i sprzedaży}
    \end{figure}

    \subsection{Implementacja innego wskaźnika}
    W celu porównania wskaźnika MACD z innym wskaźnikiem, został zaimplementowany wskaźnik

    \section{Podsumowanie}
    Tutaj pojawi się podsumowanie.

\end{document}
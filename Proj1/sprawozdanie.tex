\documentclass{article}
\usepackage[T1]{fontenc}
\usepackage{lmodern}
\usepackage[polish]{babel}
\usepackage{graphicx}
\usepackage{float}
\usepackage{hyperref}

\usepackage[a4paper, margin=2.54cm]{geometry}


\title{Sprawozdanie - Projekt 1\\Wskaźnik giełdowy MACD}
\author{Damian Jankowski s188597}
\date{06 kwietnia 2023}

\begin{document}

    \maketitle

    \section{Wstęp}
    \subsection{Czym jest wskaźnik MACD?}

    Wskaźnik \textbf{MACD} (ang. \textit{Moving Average Convergence Divergence}) 
    jest jednym z najpopularniejszych wskaźników analizy technicznej, badającym
    zależności pomiędzy krótkimi i długimi średnimi ruchomymi.
    Korzystają z niego zarówno inwestorzy indywidualni, 
    jak i instytucjonalni, a także banki i inwestycyjne domy maklerskie. \cite{macd}

    \subsection{Jak wyznaczać MACD?}

    Wskaźnik kreślony jest za pomocą dwóch wykresów:
    \begin{itemize}
        \item \textbf{MACD}
        \item \textbf{SIGNAL}
    \end{itemize}

    Obie z nich wykorzystują tzw. wykładniczą średnią ruchomą/kroczącą
    \textbf{EMA} (ang. \textit{Exponential Moving Average}), którą da się opisać
    następującą formułą \cite{ema}:
    \begin{equation}
        EMA_{N} = \frac{p_{0} + (1-\alpha)p_{1} + (1-\alpha)^2p_{2} + 
        (1-\alpha)^3p_{3} + ... + (1-\alpha)^Np_{N} }
        {1 + (1-\alpha) + (1-\alpha)^2 + (1-\alpha)^3 + ... + (1-\alpha)^N}
    \end{equation}
    gdzie:
    \begin{itemize}
        \item $EMA_{N}$ - wykładnicza średnia ruchoma z okresu $N$
        \item $p_{0}$ - aktualna cena akcji
        \item $p_{1}$ - cena akcji z poprzedniego okresu
        \item $p_{N}$ - cena akcji z $N$-tego okresu
        \item $\alpha = \frac{2}{N+1}$ - procent wykładniczy
        \item $N$ - liczba okresów wykładniczej średniej ruchomej
    \end{itemize}

    Wartość \textbf{MACD} jest wyliczana na podstawie następującej formuły:
    \begin{equation}
        MACD = EMA_{12} - EMA_{26}
    \end{equation}

    Natomiast \textbf{SIGNAL} to wykładnicza średnia ruchoma z okresu 9, policzona z \textbf{MACD}:
    \begin{equation}
        SIGNAL = EMA_{9}(MACD)
    \end{equation}

    Wynika więc z tego, że wykres \textbf{MACD} zaczyna się od 26. próbki, a \textbf{SIGNAL} od 35. próbki.

    \subsection{Jak interpretować MACD?}

    Przyjęto następujące zasady interpretacji:
    \begin{enumerate}
        \item Jeśli \textbf{MACD} przekroczy \textbf{SIGNAL} z dołu oznacza, że akcje powinny być kupowane.
        \item Jeśli \textbf{MACD} przekroczy \textbf{SIGNAL} z góry oznacza, że akcje powinny być sprzedawane.
    \end{enumerate}

    \section{Analiza wskaźnika MACD}
    \subsection{Cel analizy}

    Celem analizy było sprawdzenie, czy wskaźnik MACD jest dobrym wskaźnikiem do inwestowania w akcje
    oraz czy jest on w stanie przewidywać zmiany cen akcji w zależności od sytuacji na rynku.
    \subsection{Dane wejściowe}

    Dane wejściowe pobrałem z serwisu \textbf{Yahoo! Finance API} \cite{api} w formacie \textit{.csv} i obejmują
    ceny akcji firmy \textbf{CD Projekt Red} od 2019-01-16 do 2022-11-30.
    Był to dość ciężki okres, gdyż w tym czasie wydano grę
    \textbf{Cyberpunk 2077}, która posiadała wielkie problemy z optymalizacją podczas premiery,
    co znacząco wpłynęło na rynek.

    \begin{figure}[H]
        \includegraphics[width=\textwidth]{wykres.png}
        \centering
        \caption{Wykres ceny akcji firmy CD Projekt Red}
    \end{figure}

    Dla wybranego przedziału czasu istnieje dokładnie 1000 próbek.

    \subsection{Implementacja}

    W celu implementacji wskaźnika napisałem skrypt w języku \textit{Python}.
    Znajduje się on w pliku \textit{main.py} oraz \textit{functions.py}. Do tworzenia wykresów
    użyłem modułu \textit{matplotlib}, natomiast do wczytywania danych z API -- modułu \textit{pandas}.
    \subsection{Symulacja}
    W celu sprawdzenia poprawności działania wskaźnika przeprowadziłem dwie symulacje:
    \begin{enumerate}
        \item \textbf{Prosta} - kupowanie i sprzedawanie akcji za każdym razem, gdy wykres \textbf{MACD} przecina \\ \textbf{SIGNAL}.
        \item \textbf{Złożona} - wykorzystanie dodatkowo wskaźnika \textbf{\%R Williamsa} (ang. \textit{Williams \%R}).
    \end{enumerate}
   
    Obie symulacje zostały przeprowadzone na tych samych danych wejściowych. Kapitał początkowy wynosił 1000 złotych. 

    \subsubsection{Symulacja prosta}

    Zgodnie z przyjętymi założeniami w momencie, gdy \textbf{MACD} przekroczył \textbf{SIGNAL} z dołu, kupowano maksymalnie możliwą ilość akcji 
    za ówczesną cenę.
    Natomiast, gdy \textbf{MACD} przekroczył \textbf{SIGNAL} z góry, sprzedawano wszystkie akcje.
    
    \begin{figure}[H]
        \includegraphics[width=\textwidth]{macdsignal.png}
        \centering
        \caption{Wykres \textbf{MACD} oraz \textbf{SIGNAL} dla prostej symulacji}
    \end{figure}

    \begin{figure}[H]
        \includegraphics[width=\textwidth]{simulation_simple.png}
        \centering
        \caption{Ilość gotówki w trakcie prostej symulacji}
    \end{figure}

    W trakcie symulacji algorytm drukował informacje o aktualnie podejmowanej decyzji.
    Dla wybranej polityki inwestycyjnej, po zakończeniu symulacji, algorytm zanotował następujący zysk:
    \begin{verbatim}
        ...
        Dzień 980:  Kupiłem 8 akcje za 1089.760008 PLN.
        Dzień 992:  Sprzedałem 8 akcje za 1114.239992 PLN.
        Zysk: 151.0900469999999 PLN. Pozostało akcji: 0 sztuk.
        Sprzedaję resztę akcji za 0.0 PLN.
        Zysk całkowity: 151.0900469999999 PLN.
    \end{verbatim}

    \subsubsection{Symulacja złożona}

    W celu zwiększenia skuteczności algorytmu, zastosowałem dodatkowo wskaźnik \textbf{\%R Williamsa} \cite{wiki1}\cite{williams}.

    Używany w analizie technicznej wskaźnik \textbf{\%R} (ang. \textit{Williams \%R})
    pokazuje stosunek aktualnej ceny względem maksymalnej i minimalnej ceny w określonym okresie czasu.

    \begin{equation}
        \%R = \frac{C - H}{H - L} \cdot 100
    \end{equation}
    gdzie:
    \begin{itemize}
        \item $H$ - maksymalna cena w określonym $n$ okresie czasu
        \item $L$ - minimalna cena w określonym $n$ okresie czasu
        \item $C$ - aktualna cena
    \end{itemize}

    Wskaźnik ten przyjmuje wartości od -100 do 0.
    Wartości poniżej -80 oznaczają sprzedawanie akcji,
    natomiast wartości powyżej -20 oznaczają, że powinno się kupować akcje.

    Dla symulacji przyjąłem wartość $n = 10$.

    \begin{figure}[H]
        \includegraphics[width=\textwidth]{williams.png}
        \centering
        \caption{Wykres \textbf{\%R Williamsa} dla złożonej symulacji}
    \end{figure}

    \begin{figure}[H]
        \includegraphics[width=\textwidth]{simulation_advanced.png}
        \centering
        \caption{Ilość gotówki w trakcie złożonej symulacji}
    \end{figure}

Fragment wydruku z symulacji złożonej:
    \begin{verbatim}
        ...
        Dzień 804:  Sprzedałem 9 akcje za 1521.0 PLN.
        Dzień 943:  Kupiłem 17 akcje za 1492.6000510000001 PLN.
        Zysk: -967.8000589999999 PLN. Pozostało akcji: 17 sztuk.
        Sprzedaję resztę akcji za 2226.659932 PLN.
        Zysk całkowity: 1258.8598730000003 PLN.
    \end{verbatim}

    \subsection{Wnioski}

    Gdy przyjrzeć się fragmentowi wykresu \textbf{MACD} oraz \textbf{SIGNAL}, można zauważyć, że
    algorytm podejmował dobre decyzje, jednak w pewnych momentach nie był w stanie przewidzieć
    zmian cen akcji odpowiednio szybko. W szczególności w momencie wydania gry \textbf{Cyberpunk 2077}.

    \begin{figure}[H]
        \includegraphics[width=\textwidth]{macdsignal_zoom.png}
        \centering
        \caption{Fragment wykresu \textbf{MACD} oraz \textbf{SIGNAL}}{ wraz z zaznaczonymi momentami kupna i sprzedaży
        dla symulacji prostej}
        \label{fig:macdsignal_zoom}
    \end{figure}

    Fragment wydruku algorytmu dotyczący powyższego wykresu:
    \begin{verbatim}
        ...
        Dzień 469:  Kupiłem 4 akcje za 1460.0 PLN.
        Dzień 485:  Sprzedałem 4 akcje za 1456.0 PLN.
        Dzień 488:  Kupiłem 3 akcje za 1168.5 PLN.
        Dzień 496:  Sprzedałem 3 akcje za 1086.0 PLN.
        Dzień 509:  Kupiłem 5 akcje za 1402.5 PLN.
        Dzień 537:  Sprzedałem 5 akcje za 1466.9999699999998 PLN.
        Dzień 562:  Kupiłem 6 akcje za 1359.0 PLN.
        Dzień 575:  Sprzedałem 6 akcje za 1143.0 PLN.
        ...
    \end{verbatim}

    Wynika to z tego, że oba te wskaźniki opierają się na stosunkowo dużym
    okresie czasu wstecz. W przypadku \textbf{MACD} dopiero po 26 próbkach, a w przypadku \textbf{SIGNALA}
    dopiero po 35 próbkach jesteśmy w stanie dokonywać analizy. Stąd nadaje się
    on raczej do inwestycji długoterminowych, niż krótkoterminowych.

    Z kolei wskaźnik \textbf{\%R Williamsa} jest w stanie przewidzieć zmiany cen
    korzystając z mniejszej ilości danych.

    \section{Podsumowanie}
    Wskaźnik \textbf{MACD} dla wybranych przeze mnie danych wykazał się dobrą skutecznością i moim
    zdaniem jest przydatny w analizie technicznej.
    W przypadku obu symulacji algorytm zwrócił zysk, choć wariant złożony osiągnął lepszy wynik.

    Prosta symulacja oparta tylko o \textbf{MACD} zwróciła zysk w wysokości 151 PLN.
    Patrząc na wykres \textbf{MACD} oraz \textbf{SIGNAL} można zauważyć, że algorytm
    podejmował poprawne decyzje. Dokładnie to widać na Rysunku \ref{fig:macdsignal_zoom}.

    Natomiast złożona symulacja, która uwzględniała również \textbf{\%R Williamsa} 
    zwróciła zysk w wysokości 1258 PLN. Dodanie kolejnych warunków spowodowało, że
    algorytm podejmował mniej, ale za to bardziej trafnych decyzji.

    \renewcommand{\refname}{Źródła}
    \begin{thebibliography}{100}
     \bibitem{macd} MACD -- \url{https://pl.wikipedia.org/wiki/MACD}
     \bibitem{ema} Średnia ruchoma EMA -- \url{https://pl.wikipedia.org/wiki/%C5%9Arednia_ruchoma}
    \bibitem{api} Dane wejściowe -- \url{https://finance.yahoo.com/}
    \bibitem{wiki1} Wskaźniki analizy technicznej -- \url{https://pl.wikipedia.org/wiki/Wska%C5%BAniki_analizy_technicznej}
    \bibitem{williams} \%R Williamsa -- \url{https://pl.wikipedia.org/wiki/%25R_Williamsa}
    \bibitem{} Investopedia -- \url{https://www.investopedia.com/terms/m/macd.asp}
    \end{thebibliography}



\end{document}